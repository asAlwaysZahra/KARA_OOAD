\documentclass[12pt]{article}

\usepackage{hyperref}
\usepackage{graphicx}
\usepackage{enumitem}
\usepackage{pifont}
\usepackage{longtable}
\usepackage{xepersian}

\settextfont{Vazirmatn}
\linespread{1.25}

\author{متین اعظمی}
\author{محمد حسینی}
\author{عسل خائف}
\author{ارشیا شفیعی}
\author{شیدا عابدپور}
\author{امیرعلی لطفی}
\author{زهرا معصومی}
\title{سامانه کارا}

\begin{document}
	
	\section{برنامه}
	
	% Please add the following required packages to your document preamble:
	% \usepackage[table,xcdraw]{xcolor}
	% If you use beamer only pass "xcolor=table" option, i.e. \documentclass[xcolor=table]{beamer}
	\begin{table}[]
		\begin{tabular}{|r|r|}
			\hline
			\multicolumn{2}{|r|}{\textbf{UC9: مشاهده حقوق تخمین زده شده از ماشین حساب حقوق}}                                                                                                                                                                                           \\ \hline
			\multicolumn{2}{|r|}{\textbf{پیش شرط: -}}                                                                                                                                                                                                                                                          \\ \hline
			\multicolumn{1}{|c|}{\textbf{کنشگر: کاربر بازدیدکننده}}                                                                                       & \multicolumn{1}{c|}{\textbf{سیستم:سامانه کارا}}                                                                                                    \\ \hline
			\multicolumn{1}{|c|}{\textbf{}}                                                                                                               & \multicolumn{1}{r|}{۰-  سیستم صفحه اصلی را نشان می‌دهد.}                                                                                           \\ \hline
			\multicolumn{1}{|r|}{\begin{tabular}[c]{@{}r@{}}۱- TUCBW کاربر بازدیدکننده در صفحه  روی پیوند \\ «ماشین حساب حقوق» کلیک می‌کند.\end{tabular}} & \multicolumn{1}{r|}{\begin{tabular}[c]{@{}r@{}}۲- سیستم دو گزینه «ثبت حقوق دریافتی» \\ و «تخمین حقوق» را نمایش می‌دهد.\end{tabular}}               \\ \hline
			\multicolumn{1}{|r|}{\begin{tabular}[c]{@{}r@{}}۳- کاربر بازدیدکننده روی گزینه\\  «تخمین حقوق» کلیک می‌کند.\end{tabular}}                     & \multicolumn{1}{r|}{\begin{tabular}[c]{@{}r@{}}۴- سیستم فرمی شامل اطلاعات عنوان شغلی،\\  سطح ارشدیت، سابقه کاری و … را نمایش می‌دهد.\end{tabular}} \\ \hline
			\multicolumn{1}{|r|}{\begin{tabular}[c]{@{}r@{}}۵- کاربر بازدیدکننده اطلاعات خود را وارد می‌کند \\ سپس روی دکمه «تخمین» کلیک می‌کند.\end{tabular}}                  & \begin{tabular}[c]{@{}r@{}}۶- سیستم براساس اطلاعات موجود \\ یک حقوق تخمین زده شده را نمایش می‌دهد.\end{tabular}                                    \\ \hline
			\multicolumn{1}{|r|}{\begin{tabular}[c]{@{}r@{}}۷- TUCEW کاربر بازدیدکننده حقوق تخمین زده شده\\  را مشاهده می‌کند.\end{tabular}}                                    & \multicolumn{1}{|l|}{}                                                                                                                              \\ \hline
		\end{tabular}
	\end{table}
	
\end{document}
