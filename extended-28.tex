\documentclass[12pt]{article}

\usepackage{hyperref}
\usepackage{graphicx}
\usepackage{enumitem}
\usepackage{pifont}
\usepackage{longtable}
\usepackage{xepersian}

\settextfont{Vazirmatn}
\linespread{1.25}

\author{متین اعظمی}
\author{محمد حسینی}
\author{عسل خائف}
\author{ارشیا شفیعی}
\author{شیدا عابدپور}
\author{امیرعلی لطفی}
\author{زهرا معصومی}
\title{سامانه کارا}

\begin{document}
	
	\section{برنامه}
	% Please add the following required packages to your document preamble:
	% \usepackage[table,xcdraw]{xcolor}
	% If you use beamer only pass "xcolor=table" option, i.e. \documentclass[xcolor=table]{beamer}
	\begin{table}[]
		\begin{tabular}{|rr|}
			\hline
			\multicolumn{2}{|r|}{\textbf{UC28: ثبت آگهی}}                                                                                                                                                                                                                                                 \\ \hline
			\multicolumn{2}{|r|}{\textbf{پیش شرط: کارفرما باید وارد حساب کاربری خود شده باشد.}}                                                                                                                                                                                                                                   \\ \hline
			\multicolumn{1}{|c|}{\textbf{کنشگر: کارفرما}}                                                                                          & \multicolumn{1}{c|}{\textbf{سیستم: سامانه کارا}}                                                                                                                             \\ \hline
			\multicolumn{1}{|l|}{\textbf{}}                                                                                                        & ۰- سیستم صفحه کاربری کارفرما را نمایش می‌دهد.                                                                                                                                \\ \hline
			\multicolumn{1}{|r|}{\begin{tabular}[c]{@{}r@{}}۱- TUCBW کارفرما روی گزینه «ثبت آگهی» در\\  صفحه کاربری خود کلیک می‌کند.\end{tabular}} & \begin{tabular}[c]{@{}r@{}}۲- سیستم صفحه‌ی ساخت آگهی را که شامل فرمی \\ از اطلاعاتی مانند عنوان شغلی، نوع همکاری، \\ بازه حقوق، سطح ارشدیت و … را نمایش می‌دهد.\end{tabular} \\ \hline
			\multicolumn{1}{|r|}{\begin{tabular}[c]{@{}r@{}}۳- کاربر اطلاعات خود را وارد می‌کند \\ و سپس روی دکمه «ثبت» کلیک می‌کند.\end{tabular}} & \begin{tabular}[c]{@{}r@{}}۴- سیستم در صورت معتبر بودن اطلاعات، \\ آگهی را می‌سازد و نتیجه را به کاربر نمایش می‌دهد.\end{tabular}                                            \\ \hline
			\multicolumn{1}{|r|}{\begin{tabular}[c]{@{}r@{}}۵- TUCEW کارفرما نتیجه ساخت آگهی خود\\  را مشاهده می‌کند.\end{tabular}}                & \multicolumn{1}{l|}{}                                                                                                                                                        \\ \hline
		\end{tabular}
	\end{table}
	
\end{document}
