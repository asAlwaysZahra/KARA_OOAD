\documentclass[12pt]{article}

\usepackage{hyperref}
\usepackage{graphicx}
\usepackage{enumitem}
\usepackage{pifont}
\usepackage{longtable}
\usepackage{xepersian}

\settextfont{Vazirmatn}
\linespread{1.25}

\author{متین اعظمی}
\author{محمد حسینی}
\author{عسل خائف}
\author{ارشیا شفیعی}
\author{شیدا عابدپور}
\author{امیرعلی لطفی}
\author{زهرا معصومی}
\title{سامانه کارا}

\begin{document}
	
	\section{برنامه}
	% Please add the following required packages to your document preamble:
	% \usepackage[table,xcdraw]{xcolor}
	% If you use beamer only pass "xcolor=table" option, i.e. \documentclass[xcolor=table]{beamer}
	\begin{table}[]
		\begin{tabular}{|rr|}
			\hline
			\multicolumn{2}{|r|}{\textbf{UC24: ساخت رزومه}}                                                                                                                                                                                                                                                                                           \\ \hline
			\multicolumn{2}{|r|}{\textbf{پیش‌ شرط: کارجو باید اطلاعات کاربری خود را تکمیل کرده باشد.}}                                                                                                                                                                                                                                                                        \\ \hline
			\multicolumn{1}{|c|}{\textbf{کنشگر: کارجو}}                                                                                                             & \multicolumn{1}{c|}{\textbf{سیستم: سامانه کارا}}                                                                                                                                                        \\ \hline
			\multicolumn{1}{|l|}{\textbf{}}                                                                                                                         & ۰- سیستم صفحه کاربری کارجو را نمایش می‌دهد.                                                                                                                                                             \\ \hline
			\multicolumn{1}{|r|}{\begin{tabular}[c]{@{}r@{}}۱- TUCBW کارجو بر روی گزینه "ساخت رزومه" \\ در صفحه کاربری خود کلیک می‌کند.\end{tabular}}               & \begin{tabular}[c]{@{}r@{}}۲- سیستم صفحه ساخت رزومه که شامل فرمی \\ از اطلاعات مانند درباره من، اطلاعات تحصیلی، \\ سوابق شغلی، مهارت‌ها و … را به دو\\  زبان انگلیسی و فارسی نمایش می‌دهد.\end{tabular} \\ \hline
			\multicolumn{1}{|r|}{\begin{tabular}[c]{@{}r@{}}۳- کاربر اطلاعات خود را به فارسی یا \\ انگلیسی وارد می‌کند و روی دکمه\\  ثبت کلیک می‌کند.\end{tabular}} & \begin{tabular}[c]{@{}r@{}}۴- سیستم نتیجه ساخت رزومه \\ را به کاربر نمایش می‌دهد.\end{tabular}                                                                                                          \\ \hline
			\multicolumn{1}{|r|}{\begin{tabular}[c]{@{}r@{}}۵- TUCEW کارجو نتیجه ساخت رزومه \\ خود را مشاهده می‌کند.\end{tabular}}                                  & \multicolumn{1}{l|}{}                                                                                                                                                                                   \\ \hline
		\end{tabular}
	\end{table}
	
\end{document}
