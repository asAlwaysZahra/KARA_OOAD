\documentclass[12pt]{article}

\usepackage{hyperref}
\usepackage{graphicx}
\usepackage{enumitem}
\usepackage{pifont}
\usepackage{longtable}
\usepackage{xepersian}

\settextfont{Vazirmatn}
\linespread{1.25}

\author{متین اعظمی}
\author{محمد حسینی}
\author{عسل خائف}
\author{ارشیا شفیعی}
\author{شیدا عابدپور}
\author{امیرعلی لطفی}
\author{زهرا معصومی}
\title{سامانه کارا}

\begin{document}
	
	\section{برنامه}
	% Please add the following required packages to your document preamble:
	% \usepackage[table,xcdraw]{xcolor}
	% If you use beamer only pass "xcolor=table" option, i.e. \documentclass[xcolor=table]{beamer}
	\begin{table}[]
		\begin{tabular}{|rr|}
			\hline
			\multicolumn{2}{|r|}{\textbf{UC18: ارسال درخواست شغلی}}                                                                                                                                                                                                                                                                                                                                                    \\ \hline
			\multicolumn{2}{|r|}{\textbf{پیش شرط: کارجو باید وارد حساب کاربری خود شده باشد و رزومه ساخته باشد.}}                                                                                                                                                                                                                                                                                                                               \\ \hline
			\multicolumn{1}{|c|}{\textbf{کنشگر: کارجو}}                                                                                                                                        & \multicolumn{1}{c|}{\textbf{سیستم:سامانه کارا}}                                                                                                                                                                                               \\ \hline
			\multicolumn{1}{|l|}{\textbf{}}                                                                                                                                                    & ۰- سیستم صفحه آگهی را به کارجو نمایش می‌دهد.                                                                                                                                                                                                  \\ \hline
			\multicolumn{1}{|r|}{\begin{tabular}[c]{@{}r@{}}۱- TUCBW کارجو روی گزینه «ارسال \\ درخواست شغلی» در صفحه آگهی کلیک می‌کند.\end{tabular}}                                           & \begin{tabular}[c]{@{}r@{}}۲- سیستم صفحه تایید اطلاعات و پیش شرط‌های\\  لازم مانند تست شخصیتی و فرم پرسشنامه\\  برای ارسال درخواست را به کارجو نمایش می‌دهد.\end{tabular}                                                                     \\ \hline
			\multicolumn{1}{|r|}{\begin{tabular}[c]{@{}r@{}}۵- در صورت وجود فرم پرسشنامه، کاربر\\  آن را تکمیل می‌کند.در نهایت روی گزینه \\ "تایید و ارسال درخواست" کلیک می‌کند.\end{tabular}} & \begin{tabular}[c]{@{}r@{}}۶- \\ الف) در صورتی که درصد تطابق رزومه و آگهی\\  بالاتر از ۵۰ درصد بود سیستم پیام \\ "ارسال درخواست شغلی با موفقیت انجام شد." \\  را نمایش می‌دهد.\\ ب) در غیر این صورت سیستم پیام خطا نمایش می‌دهد.\end{tabular} \\ \hline
			\multicolumn{1}{|r|}{\begin{tabular}[c]{@{}r@{}}۷-TUCEW کارجو نتیجه ارسال درخواست\\  شغلی خود را در قالب پیام مناسب \\ مشاهده می‌کند.\end{tabular}}                                & \multicolumn{1}{l|}{}                                                                                                                                                                                                                         \\ \hline
		\end{tabular}
	\end{table}
	
\end{document}
