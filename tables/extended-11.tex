\documentclass[12pt]{article}

\usepackage{hyperref}
\usepackage{graphicx}
\usepackage{enumitem}
\usepackage{pifont}
\usepackage{longtable}
\usepackage{xepersian}

\settextfont{Vazirmatn}
\linespread{1.25}

\author{متین اعظمی}
\author{محمد حسینی}
\author{عسل خائف}
\author{ارشیا شفیعی}
\author{شیدا عابدپور}
\author{امیرعلی لطفی}
\author{زهرا معصومی}
\title{سامانه کارا}

\begin{document}
	
	\section{برنامه}
	
	% Please add the following required packages to your document preamble:
	% \usepackage[table,xcdraw]{xcolor}
	% If you use beamer only pass "xcolor=table" option, i.e. \documentclass[xcolor=table]{beamer}
	\begin{table}[]
		\begin{tabular}{|rr|}
			\hline
			\multicolumn{2}{|r|}{\textbf{UC11: ارسال پیام خصوصی}}                                                                                                                                                                                                                                                                                                                                        \\ \hline
			\multicolumn{2}{|r|}{\textbf{پیش شرط: کاربر باید وارد حساب کاربری خود شده باشد.}}                                                                                                                                                                                                                                                                                                                                    \\ \hline
			\multicolumn{1}{|c|}{\textbf{کنشگر:کاربر}}                                                                                                                                                                                                                    & \multicolumn{1}{c|}{\textbf{سیستم:سامانه کارا}}                                                                                                      \\ \hline
			\multicolumn{1}{|l|}{\textbf{}}                                                                                                                                                                                                                               & ۰- سیستم صفحه اصلی را نمایش می‌دهد.                                                                                                                  \\ \hline
			\multicolumn{1}{|r|}{\begin{tabular}[c]{@{}r@{}}۱- TUCBW کاربر روی پیوند "پیام خصوصی"\\  در صفحه اصلی کلیک می‌کند.\end{tabular}}                                                                                                                              & ۲- سیستم صفحه پیام‌رسان را نمایش می‌دهد.                                                                                                             \\ \hline
			\multicolumn{1}{|r|}{\begin{tabular}[c]{@{}r@{}}۳-\\ الف) اگر این اولین پیام خصوصی به کاربر مورد نظر \\ است، کاربر ایمیل شخص مورد نظر را وارد می‌کند.\\  ب) در غیر این صورت، کاربر مورد نظر را از\\ لیست گفتگوها برای ارسال پیام انتخاب می‌کند.\end{tabular}} & \begin{tabular}[c]{@{}r@{}}۴- در صورت موجود بودن شخص، صفحه‌ی گفتگو \\ شخص مورد نظر به همراه فیلد شرح پیام و پیوست \\ نمایش داده می‌شود.\end{tabular} \\ \hline
			\multicolumn{1}{|r|}{\begin{tabular}[c]{@{}r@{}}۵- کاربر صفحه گفتگو را مشاهده می‌کند و \\ متن پیام خود را وارد می‌کند. در نهایت روی دکمه \\ "ارسال" کلیک می‌کند.\end{tabular}}                                                                                & \begin{tabular}[c]{@{}r@{}}۶- سیستم پیامی متناسب با نتیجه\\  ارسال پیام نمایش می‌دهد.\end{tabular}                                                   \\ \hline
			\multicolumn{1}{|r|}{\begin{tabular}[c]{@{}r@{}}۷- TUCEW کاربر نتیجه ارسال پیام خصوصی \\ خود را مشاهده می‌کند.\end{tabular}}                                                                                                                                  & \multicolumn{1}{l|}{}                                                                                                                                \\ \hline
		\end{tabular}
	\end{table}
	
\end{document}
